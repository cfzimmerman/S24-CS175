\documentclass[letterpaper, 11pt]{article}
\usepackage[margin=1in]{geometry}
\usepackage{graphicx}
\usepackage{amssymb}
\usepackage{epstopdf}
\usepackage{url}
\usepackage{hyperref}
\usepackage{graphicx}
\usepackage{amsmath}

\newcommand{\ft}{$\vec{f}^t$}
\newcommand{\fr}[1]{\vec{#1}^t}


\begin{document}
\noindent \textit{Cory Zimmerman}
\medskip

\noindent \textbf{(Q1)} \\ 
TODO

\medskip
\noindent \textbf{(Q2: 11.3)} \\ 
TODO

\medskip
\noindent \textbf{(Q3: 11.2)} \\ 
Let the projection matrix be defined as \\ \\ 
$P = \begin{bmatrix}
1 & 0 & 0 & 0 \\
0 & 1 & 0 & 0 \\
0 & 0 & 0 & 1 \\
0 & 0 & -1 & 0
\end{bmatrix}$ \\

\noindent With eye coordinates $x_e, y_e, z_e$, this yields clip coordinates \\ \\ 
$\begin{bmatrix}
1 & 0 & 0 & 0 \\
0 & 1 & 0 & 0 \\
0 & 0 & 0 & 1 \\
0 & 0 & -1 & 0
\end{bmatrix} 
\cdot 
\begin{bmatrix}
x_e  \\
y_e  \\
z_3  \\
1 
\end{bmatrix}
=
\begin{bmatrix}
x_e  \\
y_e  \\
1 \\
-z_3
\end{bmatrix}$ \\ \\ 

\noindent This produces device coordinates \\ \\ 
$\begin{bmatrix}
\frac{x_e}{-z_e}  \\
\frac{y_e}{-z_e}  \\
\frac{1}{-z_e} \\
1
\end{bmatrix}$ \\ \\ 

Using the projection matrix $PQ$ instead generates \\ \\ 
$PQ = \begin{bmatrix}
1 & 0 & 0 & 0 \\
0 & 1 & 0 & 0 \\
0 & 0 & 0 & 1 \\
0 & 0 & -1 & 0
\end{bmatrix} 
\begin{bmatrix}
3 & 0 & 0 & 0 \\
0 & 3 & 0 & 0 \\
0 & 0 & 3 & 1 \\
0 & 0 & 0 & 1
\end{bmatrix}
=
\begin{bmatrix}
3 & 0 & 0 & 0 \\
0 & 3 & 0 & 0 \\
0 & 0 & 0 & 1 \\
0 & 0 & -3 & 0
\end{bmatrix},
PQ \cdot 
\begin{bmatrix}
x_e  \\
y_e  \\
z_3  \\
1 
\end{bmatrix}
=
\begin{bmatrix}
3x_e  \\
3y_e  \\
1  \\
-3z_e 
\end{bmatrix}$ \\ \\ 


\noindent This produces device coordinates \\ \\ 
$\begin{bmatrix}
\frac{3x_e}{-3z_e}  \\
\frac{3y_e}{-3z_e}  \\
\frac{1}{-3z_e} \\
1
\end{bmatrix}
=
\begin{bmatrix}
\frac{x_e}{-z_e}  \\
\frac{y_e}{-z_e}  \\
\frac{1}{-3z_e} \\
1
\end{bmatrix}$ \\ \\ 

Mathematically, it appears that the coordinates projected by $PQ$ have identical $x$ and $y$ values but are squished closer towards the origin in the $z$ direction compared to coordinates projected by just $P$. I believe this has the effect of extending the $z$-length of the visual frustrum, fitting more total coordinates into the view window. In a rendering environment, I think this would make us see further.

\medskip
I made a test example in \texttt{mtx\_proj.py}. With eye coordinates $(0, 0,-1, 1), (0.5, 0, -0.5, 1), (0, 0.5, 0.5, 1)$, the device coordinates projected by just $P$ are $(0, 0, 1, 1), (1, 0, 2, 1), (0, -1, -2, 1)$, while the device coordinates projected by $PQ$ are $(0, 0, 0.33, 1), (1, 0, 0.67, 1), (0, -1, -0.67, 1)$. 
\end{document}
