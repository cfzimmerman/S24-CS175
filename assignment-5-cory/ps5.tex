\documentclass[letterpaper, 11pt]{article}
\usepackage[margin=1in]{geometry}
\usepackage{graphicx}
\usepackage{amssymb}
\usepackage{epstopdf}
\usepackage{url}
\usepackage{hyperref}
\usepackage{graphicx}
\usepackage{amsmath}
\usepackage{hyperref}
\usepackage{parskip}

\newcommand{\ft}{$\vec{f}^t$}
\newcommand{\fr}[1]{\vec{#1}^t}


\begin{document}
\noindent \textit{Cory Zimmerman}
\medskip

\noindent \textbf{(Q1)} \\ 
TODO

\medskip
\textbf{(Q3: 11.2)} \\ 
\textit{Notice that this is Q3. I'm doing them in book order, not pset order}: \\
Let the projection matrix be defined as \\ \\ 
$P = \begin{bmatrix}
1 & 0 & 0 & 0 \\
0 & 1 & 0 & 0 \\
0 & 0 & 0 & 1 \\
0 & 0 & -1 & 0
\end{bmatrix}$ \\

With eye coordinates $x_e, y_e, z_e$, this yields clip coordinates \\ \\ 
$\begin{bmatrix}
1 & 0 & 0 & 0 \\
0 & 1 & 0 & 0 \\
0 & 0 & 0 & 1 \\
0 & 0 & -1 & 0
\end{bmatrix} 
\cdot 
\begin{bmatrix}
x_e  \\
y_e  \\
z_3  \\
1 
\end{bmatrix}
=
\begin{bmatrix}
x_e  \\
y_e  \\
1 \\
-z_3
\end{bmatrix}$ \\ \\ 

This produces device coordinates \\ \\ 
$\begin{bmatrix}
\frac{x_e}{-z_e}  \\
\frac{y_e}{-z_e}  \\
\frac{1}{-z_e} \\
1
\end{bmatrix}$ \\ \\ 

Using the projection matrix $PQ$ instead generates \\ \\ 
$PQ = \begin{bmatrix}
1 & 0 & 0 & 0 \\
0 & 1 & 0 & 0 \\
0 & 0 & 0 & 1 \\
0 & 0 & -1 & 0
\end{bmatrix} 
\begin{bmatrix}
3 & 0 & 0 & 0 \\
0 & 3 & 0 & 0 \\
0 & 0 & 3 & 0 \\
0 & 0 & 0 & 1
\end{bmatrix}
=
\begin{bmatrix}
3 & 0 & 0 & 0 \\
0 & 3 & 0 & 0 \\
0 & 0 & 0 & 1 \\
0 & 0 & -3 & 0
\end{bmatrix},
PQ \cdot 
\begin{bmatrix}
x_e  \\
y_e  \\
z_3  \\
1 
\end{bmatrix}
=
\begin{bmatrix}
3x_e  \\
3y_e  \\
1  \\
-3z_e 
\end{bmatrix}$ \\ \\ 


This produces device coordinates \\ \\ 
$\begin{bmatrix}
\frac{3x_e}{-3z_e}  \\
\frac{3y_e}{-3z_e}  \\
\frac{1}{-3z_e} \\
1
\end{bmatrix}
=
\begin{bmatrix}
\frac{x_e}{-z_e}  \\
\frac{y_e}{-z_e}  \\
\frac{1}{-3z_e} \\
1
\end{bmatrix}$ \\ \\ 

Mathematically, it appears that the coordinates projected by $PQ$ have identical $x$ and $y$ values but are squished closer towards the origin in the $z$ direction compared to coordinates projected by just $P$. I beliee this has the effect of pushing a wider range of $z$ values into the range of what will be rendered, producing the effect of seeing further.

\medskip
I made a test example available here: \url{https://github.com/cfzimmerman/S24-CS175/blob/main/assignment-5-cory/mtx_proj.py}. With eye coordinates $(0, 0,-1, 1), (0.5, 0, -0.5, 1), (0, 0.5, 0.5, 1)$, the device coordinates projected by just $P$ are $(0, 0, 1, 1), (1, 0, 2, 1), (0, -1, -2, 1)$, while the device coordinates projected by $PQ$ are $(0, 0, 0.33, 1), (1, 0, 0.67, 1), (0, -1, -0.67, 1)$. 

\medskip
\textbf{(Q2: 11.3)} \\ 
Abbreviating some of the steps from the question above, consider the projection by $PS$: \\ \\
$PQ = \begin{bmatrix}
1 & 0 & 0 & 0 \\
0 & 1 & 0 & 0 \\
0 & 0 & 0 & 1 \\
0 & 0 & -1 & 0
\end{bmatrix} 
\begin{bmatrix}
3 & 0 & 0 & 0 \\
0 & 3 & 0 & 0 \\
0 & 0 & 3 & 0 \\
0 & 0 & 0 & 3
\end{bmatrix}
=
\begin{bmatrix}
3 & 0 & 0 & 0 \\
0 & 3 & 0 & 0 \\
0 & 0 & 0 & 3 \\
0 & 0 & -3 & 0
\end{bmatrix},
PQ \cdot 
\begin{bmatrix}
x_e  \\
y_e  \\
z_e  \\
1 
\end{bmatrix}
=
\begin{bmatrix}
3x_e  \\
3y_3 \\
3 \\
-3z_e \\ 
\end{bmatrix}$ \\ \\ 

This produces device coordinates \\ \\ 
$\begin{bmatrix}
  \frac{3x_e}{-3z_e}  \\
  \frac{3y_e}{-3z_e} \\
  \frac{3}{-3z_e} \\
  \frac{-3z_e}{-3z_e} \\ 
\end{bmatrix}
=
\begin{bmatrix}
  \frac{x_e}{-z_e}  \\
  \frac{y_e}{-z_e} \\
  \frac{1}{-z_e} \\
  1 \\ 
\end{bmatrix}$ \\ \\ 

The resulting device coordinates are identical to those after projection by just $P$. This shows up mathematically because all entries in the output are scaled by $3$, and then dividing by $w_c$ eliminates that scale from all entries. Visually, I believe this has the effect of expanding evenly expanding and then condensing our view of the world by the same constant, producing the same picture. I added code for this to the script linked above as well, but, because the outputs are the same for both, they're not very interesting. 

\medskip
\textbf{(Q4: 11.4)} \\ 
Again, model mathematically: \\ \\ 
$QP = 
\begin{bmatrix}
3 & 0 & 0 & 0 \\
0 & 3 & 0 & 0 \\
0 & 0 & 3 & 0 \\
0 & 0 & 0 & 3
\end{bmatrix}
\begin{bmatrix}
1 & 0 & 0 & 0 \\
0 & 1 & 0 & 0 \\
0 & 0 & 0 & 1 \\
0 & 0 & -1 & 0
\end{bmatrix}
=
\begin{bmatrix}
3 & 0 & 0 & 0 \\
0 & 3 & 0 & 0 \\
0 & 0 & 0 & 3 \\
0 & 0 & -1 & 0
\end{bmatrix}
\cdot 
\begin{bmatrix}
x_e  \\
y_e  \\
z_e  \\
1 
\end{bmatrix}
=
\begin{bmatrix}
3x_e  \\
3y_e  \\
3  \\
-z_e 
\end{bmatrix}$ \\ \\

This produces device coordinates \\ \\ 
$\begin{bmatrix}
  \frac{3x_e}{-z_e} \\
  \frac{3y_e}{-z_e}  \\
  \frac{3}{-z_e}  \\
  1 
\end{bmatrix}$ \\ \\ 

Under this transformation, the $x$, $y$, and $x$ coordinates are all scaled by a factor of $3$. With these expanded dimensions, fewer points will fit within the clipped viewing region, creating a zooming effect on the scene.

\medskip
Again using the same script as before, the same eye coordinates $(0, 0,-1, 1), (0.5, 0, -0.5, 1), (0, 0.5, 0.5, 1)$ which map under just $P$ to $(0, 0, 1, 1), (1, 0, 2, 1), (0, -1, -2, 1)$ now map under $QP$ to $(0, 0, 3, 1)$, $(3, 0, 6, 1), (0, 3, 6, 1)$. 


\medskip
\textbf{(Q5: 12.3)} \\ 
todo
\end{document}
